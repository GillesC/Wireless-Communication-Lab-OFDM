%\usepackage[english]{babel}
\usepackage[english]{babel}
\usepackage[T1]{fontenc}
\usepackage{graphicx}

\usepackage{float}
\floatstyle{ruled}
\newfloat{program}{thp}{lop}
\floatname{program}{Program}

% Pakete für Mathe:
\usepackage{amsfonts,amsmath,amsthm,amscd,amssymb,array}
\usepackage{nicefrac}

\usepackage{xcolor, fancybox}
\usepackage{fix-cm}
\usepackage{lmodern}

\definecolor{defaultColor}{HTML}{DD8A2E}
\usepackage[colorlinks=true, linkcolor=defaultColor, anchorcolor=defaultColor, citecolor=defaultColor, filecolor=defaultColor, urlcolor=defaultColor]{hyperref} % Use the color defined 

% Pakete für Algorithmen
%\usepackage{algorithm,ifthen,url,algpseudocode}
\usepackage{algpseudocode,algorithm}

%Pakete zum Erstellen von ps-Grafiken
\usepackage{pstricks,pst-plot,pst-node}

% Paket, mit dem man festlegen kann, was geschehen soll, wenn die Seite voll ist.
\usepackage{afterpage}


\usepackage{marvosym}

% "Einrichten" des theorem-Paketes:
\newtheorem{thm}{Satz}[chapter]
\newtheorem{lem}[thm]{Lemma}
\newtheorem{cor}[thm]{Corollary}
\newtheorem*{nnlem}{Lemma}
\newtheorem*{nnthm}{Theorem}

\theoremstyle{definition}
\newtheorem{exmp}[thm]{Example}
\newtheorem*{nnexmp}{Example}
\newtheorem{defn}[thm]{Definition}
\newtheorem*{nndefn}{Definition}

\theoremstyle{remark}
\newtheorem{rem}[thm]{Remark}
\newtheorem*{nnrem}{Remark}

% Einige Makros:
\newcommand{\modulo}{\mbox{mod }}
\newcommand{\Z}{\mathbb{Z}}
\newcommand{\N}{\mathbb{N}}
\newcommand{\J}{\mathbb{J}}
\newcommand{\HD}{\mathbb{P}}
\newcommand{\D}{\mathbb{D}}
\newcommand{\R}{\mathbb{R}}
\newcommand{\CO}{\mathcal{O}}
\newcommand{\F}{\mathbb{F}}
\newcommand{\divisor}{\mathrm{div}}
\newcommand{\ord}{\mathrm{ord}}
\newcommand{\supp}{\mathrm{supp}}
\newcommand{\ggt}{\mathrm{ggt}}
\newcommand{\charac}{\mathrm{char}}
\renewcommand{\algorithmicrequire}{\textbf{Input:}}
\renewcommand{\algorithmicensure}{\textbf{Output:}}
%\renewcommand{\listalgorithmname}{Algorithms}
\makeatletter
\renewcommand{\ALG@name}{Algorithm}

\newcommand{\option}{{\small -}{\small -}}
\newenvironment{bew}{\begin{proof}[Proof]}{\end{proof}}
% Komfortabler Sachen in den Index aufnehmen:
\newcommand{\idx}[1]{#1\index{#1}}

\newcommand{\bl}{\color{blue}}
\newcommand{\bk}{\color{black}}
\newcommand{\rd}{\color{red}}
\newcommand{\gr}{\color{green}}

% Now defining header and footer
%\usepackage{scrpage2} % for some decent header-lines
%\clearscrheadfoot % clear style
%\lehead{\pagemark} % page left even
%\lohead{\headmark} % heading left odd
%\rohead{\pagemark} % page right odd
%\rehead{\headmark} % heading right even
%\setheadsepline{.5pt} % thickness of separating line
%\setfootsepline{.5pt} % thickness of separating line
%\lefoot{\includegraphics[height=9pt]{ti_veclogo}} % ti left even
%\rofoot{\includegraphics[height=9pt]{RWTHAACHEN_sw}} % rwth right odd

% Seitenformatierung
% \usepackage[a4paper, top=2.0cm, bottom=1.5cm, hmargin=2.5cm, includehead, includefoot, headheight=15pt]{geometry} % Flexible and complete interface to document dimensions.
\setlength{\parindent}{0cm} % Default space between columns of tabulars
\setlength{\parskip}{5pt plus 2pt minus 1pt}
\clubpenalty = 10000 % Avoid "`Schusterjungen"'
\widowpenalty = 10000 % Avoid "`Waisenkinder"'
\displaywidowpenalty = 10000 %
\tolerance 1000 % Tolerance in linebreaks
\hbadness 1000 % Number at which badness ist reported
\emergencystretch 1.5em 

\usepackage{multirow}

\makeatletter
\DeclareRobustCommand{\Cpp}
{\valign{\vfil\hbox{##}\vfil\cr
   \textsf{C\kern-.1em}\cr
   $\hbox{\fontsize{\sf@size}{0}\textbf{+\kern-0.05em+}}$\cr}%
}
\makeatother
\makeatletter
\DeclareRobustCommand{\ITpp}
{\valign{\vfil\hbox{##}\vfil\cr
   \textsf{IT\kern-.1em}\cr
   $\hbox{\fontsize{\sf@size}{0}\textbf{+\kern-0.05em+}}$\cr}%
}
\makeatother


\makeatletter
\def\blfootnote{\gdef\@thefnmark{}\@footnotetext}
\makeatother

\usepackage[outdir=./figs/]{epstopdf}

\usepackage[capitalise]{cleveref} 

\usepackage[per-mode=symbol]{siunitx}  %usage \SI{35}{\meter\squared}
\DeclareSIUnit{\dBm}{dBm}	% add SI unit "dBm"

\usepackage{cmbright}