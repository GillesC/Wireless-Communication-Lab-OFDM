\chapter{Evaluation}\glsresetall

Evaluation of this lab session will consider students' work attitude throughout the session and a final test to assess their acquisition of requisite skills and knowledge.

Students have the option to collaborate in groups of up to two. However, it's essential to note that assignments must be completed individually. This means each student should be capable of repeating the lab session independently, as their performance will be evaluated during the final test. Additionally, when the students are referred to external sources during the assignment, this must be also known for the final test.
Further instructions will be given during the lab sessions.

Things you need to know (non-exhaustive list):
\begin{itemize}
    \item What is meant by \gls{bb}?
    \item How does the B210 get \gls{iq} samples?
    \item What was the relation between the patch antenna area and the carrier frequency of the antenna?
    \item What does the S11-parameter tell you about your antenna?
    \item How does \textit{pack K bits} block work in \gls{grc}?
    \item What is the \textit{throttle} block, and when do you (not) need it?
    \item What is the difference between streams and vectors in GNU Radio?
    \item How do you calibrate a \gls{vna} and what does each step do?
    \item Which RF architecture is used by the \gls{usrp} B210?
    \item What is frequency selective fading?
    \item How is \gls{isi} mitigated in \gls{ofdm} systems?
    \item What is the relation between a N-\gls{qam} (with N the number of constellation points) and the number of bits per symbol?
    \item What technique is used in \gls{ofdm} to go from a set of parallel symbols to one \gls{ofdm} symbol?  
    \item Why does the \textit{Schmidl \& Cox Synchronization} block does not need to a-priori know the sync word symbols?
    \item What does the plateau detector do in GNU~Radio, and why is this needed?
\end{itemize}
